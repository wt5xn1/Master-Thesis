\chapter{Background}

\section{Electromagnetic Fields}
The electromagnetic fields are a collection of closely linked fields.
These fields govern the electric and magnetic interactions of charged
particles and domains. These fields can be seen in table \ref{tab:fields}

\begin{center}
    \begin{tabular}{c c c}
        \label{tab:fields}
        Field    & SI unit         & Description              \\
        \hline
        $\vb{H}$ & $1\, Am^{-1} $  & Magnetic Field           \\
        $\vb{E}$ & $1\, Vm^{-1} $  & Electric Field           \\
        $\vb{B}$ & $1\, Vsm^{-2} $ & Magnetic Flux Density    \\
        $\vb{D}$ & $1\, Asm^{-2} $ & Electric Flux Density    \\
        $\vb{J}$ & $1\, Am^{-2} $  & Electric Current Density \\
        $\rho$   & $1\, Asm^{-3} $ & Electric Charge Density
    \end{tabular}
\end{center}
These fields are described by Maxwells Equations.
In differential form for the stationary case, these are as follows:

\begin{align}
    \nabla \times \vb{H} & = \vb{J} + \frac{\partial}{\partial t}\vb{D}
    \label{eq:maxwell1}                                                 \\
    \nabla \times \vb{E} & = -\frac{\partial}{\partial t}\vb{B}
    \label{eq:maxwell2}                                                 \\
    \nabla \cdot \vb{B}  & = \vb{0}
    \label{eq:maxwell3}                                                 \\
    \nabla \cdot \vb{D}  & = \rho
    \label{eq:maxwell4}
\end{align}
Since we're dealing with measurement of magnetic fields in this thesis,
equations \ref{eq:maxwell1} and \ref{eq:maxwell3} will naturally be of
the most interest. In simple cases, the $\vb{H}, \vb{D}, \vb{E}$ and
$\vb{B}$ field obey the easy relations
\begin{align}
    \vb{B} & = \mu \vb{H}
    \label{eq:BHmap}           \\
    \vb{D} & = \epsilon \vb{E}
\end{align}
where $\mu$ is the \emph{magnetic permeability} and $U$ is the
\emph{electric permittivity} in the domain of interest.
Formally, simple cases are where the fields are located in a medium that is
linear, homogenous across its domain, invariant depending on direction, and
stationary. Since the magnetic measurements are made inside the empty aperture
of the magnet, the domain is only made up of air. Thus, equation \ref{eq:BHmap}
holds, and the magnetic permeability is the one of free space, that is
$\mu = \mu_0 = 4\pi \times 10^{-7} Hm^{-1}$. \cite[Ch.4.1-4.4]{russenschuck2011field}


\subsection{Magnetic Flux and Induction}
Magnetic flux $\Phi$ is the surface integral of the $\vb{B}$ field
along the normal vector to the surface.
Mathematically, it is defined as:
\begin{equation}
    \Phi(\area) = \iint\limits_{\area} \vb{B} \cdot \nvec\, d\area
\end{equation}
where $\area$ is the surface, and $\nvec$ is the normal vector to the surface.
We then have the following governing laws of electromagnetism for objects at rest:
\begin{align}
    U(\partial\area)      & = -\frac{d}{dt}\Phi(\area)
    \label{eq:faraday}                                 \\
    \Phi(\partial\Volume) & = 0
    \label{eq:fluxcons}
\end{align}

\begin{figure}
    \centering
    \begin{subfigure}[b]{0.4\textwidth}
        \centering
        \includegraphics[height=125pt]{figs/partialA}
        \caption{An area $\area$ and its boundary $\partial \area$. }
        \label{fig:partialA}
    \end{subfigure}
    \hfill
    \begin{subfigure}[b]{0.4\textwidth}
        \centering
        \includegraphics[height=125pt]{figs/partialV}
        \caption{A volume $\Volume$ and its surface boundary $\partial \Volume$.}
        \label{fig:partialV}
    \end{subfigure}
\end{figure}
Equation \ref{eq:faraday}, also called faradays law, describes the voltage\
$\epsilon$ induced in a length of wire $\partial\area$, enclosing an area
$\area$, when the magnetic flux $\Phi$ is changing with respect to time.
The signs of $U$ and $\Phi$ obey the right hand rule as indicated
in figure \ref{fig:partialA}.

Equation \ref{eq:fluxcons} states that the total amount of flux flowing
through the boundary $\partial\Volume$ of the volume
$\Volume$ must equal 0. \cite[Ch.4.1.1]{russenschuck2011field}

\subsection{Series decompositions of the magnetic field}
The magnetic field can be calculated in some different ways, either
directly from Maxwells equations or using Biot-Savarts law:
\begin{equation}
    \vb{B}(\vb{r}) = \frac{\mu_0}{4\pi}\int\limits_\Volume
    \frac{\nabla \times \vb{J}(\vb{r'})}{|\vb{r} - \vb{r'}|} d\Volume
\end{equation}
where $\vb{B}(\vb{r})$ is the $\vb{B}$ field at coordinate $\vb{r}$ and
$\vb{J}(\vb{r'})$ is the current distribution at coordinate $\vb{r'}$.
\cite[Ch.5.4]{russenschuck2011field}
Except for very simple geometries, the magnetic field is rarely
expressible using elementary functions. A common method is then
to express it using fourier series solutions inside a specified domain.
\cite[Ch.6]{russenschuck2011field}

Inside the aperture of a magnet, the domain is free of currents and made
up of air or vacuum. The current powering the magnet is constant,
meaning we have a constant electric field. Equation \ref{eq:maxwell1}
can then be rewritten as follows:
\begin{align}
    \begin{split}
        \nabla \times \vb{H} &= \mu_0\nabla \times \vb{B}\\
        \mu_0\nabla \times \vb{B}
        &=\vb{J} + \frac{\partial}{\partial t}\vb{D}
        \Bigg\vert_{\substack{\vb{J}=\vb{0} \\
                \frac{\partial}{\partial t}\vb{D}=\vb{0}}} \\
        &= \vb{0}
    \end{split}
\end{align}
This, along with equation \ref{eq:maxwell3} means that there
exists a magnetic scalar potential $\Psi(\vb{r})$ that satisfies
Laplace's equation
\begin{equation}
    \nabla^2\Psi = 0
\end{equation}
inside the domain.
\subsubsection{Cylindrical Coordinates}
\subsubsection{Bessel Functions}
\subsubsection{Bessel-Fourier-Fourier Series}

\section{Signal Processing}
\subsection{Filters}
\subsection{Least Squares Fitting}