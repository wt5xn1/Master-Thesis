\chapter{Discussion}
The angle estimation from the peak shifts turned out to be 
a harder problem than expected. As the arguments of the maximums of the
$B_z$ field is not easily modeled with regards to the offset, finding
a consistent model is a challenge. Some hope remains however. In figure
\ref{fig:sim-mag-fieldmap-argmax} one can see that the argmax tends to
the magnet ends for offsets further out. One could then build a model
taking the magnet length and aperture radius into consideration, and measure
the field as far from the center of the aperture as possible.

Another possibility is to add coils or Hall effect sensors that measure
the transversal $B_x$ and $B_y$ components. If the results of section 
\ref{sec:dipole-simulations} hold for real and not only simulated magnets,
this could be a very robust way of measuring the tilt-swing angle of the
solenoid. Although the transversal dipole moments are not as strong as
the $B_z$ field, the integral of the translating measurements is very
resistant to gaussian noise. This coupled with the fact the dipole
integral is invariant to potential offset alignment errors makes for
a promising measurement methodology.

One particularly interesting result with regards to the fitting
of the BFF series is that only the outermost coils measurements
were needed for a good fit. The innermost coils did not add any
more information. This fits well with the theory on partial
differential equations. A PDE may be completely determined
by its values at its boundaries. This also adds robustness to the system,
since the weaker field in the middle of the aperture can be estimated
using measurements of the stronger field closer to the solenoid edge,
from the more sensitive outer coils.